\subsection{Chronologie des versions}
\label{chrono}
Comme expliqué à la section \ref{Outils de travail}, nous avons organisé le développement de notre projet en différentes versions, dont on peut observer les dates butoires sur la figure \ref{fig:chrono}. Voici une description très générale de l'objectif de chaque version :
\begin{itemize}
\renewcommand{\labelitemi}{$\bullet$}
\item \textbf{v0.1 :} Création de l'émulateur, et premières bases du serveur graphique\footnote{Également appelé ``système de fenêtrage''.} (définition du protocole, primitives de dessin) ;
\item \textbf{v0.2 :} Bases du gestionnaire de fenêtres : gestion des événements et décoration simpliste ;
\item \textbf{v0.3 :} Renforcement du serveur graphique : gestion de plusieurs fenêtres, gestion du clavier ;
\item \textbf{v0.4 :} Renforcement du gestionnaire de fenêtres : décoration élaborée et déplacement des fenêtres, gestion du focus, gestion de la souris ;
\item \textbf{v0.5 :} Élaboration de la librairie de widgets, gestion du texte et des images ;
\item \textbf{v1.0 :} Intégration dans TacOS, création d'applications graphiques basées sur la librairie de widgets.
\end{itemize}

\begin{figure}[H]
  \centering
    \includegraphics[width=15cm]{figures/chrono}
  \caption{\label{fig:chrono}Chronologie des versions du projet}
\end{figure}

\subsection{Diagramme de Gantt}
\label{taches}
Sur le diagramme de Gantt (figure \ref{fig:diagGant}), on peut observer les dates butoires de chacune des versions, ainsi que toutes les demandes associées (de type conception et développement). Pour chaque demande, on peut connaître sa date de début, sa date de fin, son état (nouvelle, assignée, en cours, fermée...), et son pourcentage d'avancement. On observe également la répartition des tâches dans les différentes versions.
\begin{figure}[H]
  \centering
    \includegraphics[width=15cm]{figures/tacos-gui-gantt}
  \caption{\label{fig:diagGant}Diagramme de Gantt du projet}
\end{figure}

\subsection{Charge en temps selon le type des demandes}
On peut observer sur la figure \ref{fig:pie} la répartition du temps selon les types de demandes.
\begin{figure}[H]
  \centering
    \includegraphics[width=10cm]{figures/pie}
  \caption{\label{fig:pie}Répartition de la charge horaire selon les types de demandes}
\end{figure}
