\subsection{Revues hebdomadaires}

\paragraph{}
Chaque dimanche nous faisons un bilan du travail de la semaine pour faire le compte rendu à rendre aux tuteurs. Ce bilan nous permet d'y voir plus clair et de bien mettre à jour le wiki mais nous en profitons également pour vérifier le travail des autres. Nous faisons des tests sur les nouvelles fonctionnalités et nous vérifions les codes par binômes.

\subsection{Revues de version}

\paragraph{}
Avant de cloturer une version, nous vérifions que toutes les demandes ont bien été terminées. Nous effectuons donc de nouveaux tests. Ceci est important pour faire des tests de non régression. Si l'achévement du dernier ticket cause des bugs dans d'autres fonctionnalités.

\subsection{Revues de code}

\paragraph{}
Les revues de codes sont effectuées par tout le monde grâce à l'interface web redmine. Il est possible de comparer les versions (différences). On peut donc facilement voir les ajouts et suppressions effectuées par les autres membres depuis une certaine révision. Pour une personne qui fait une modification, il y a donc trois personnes qui vérifient le travail.