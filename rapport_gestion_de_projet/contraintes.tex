\definecolor{red}{rgb}{1., 0., 0.}
\definecolor{green}{rgb}{0., 1., 0.}
\definecolor{orange}{rgb}{1., 0.39, 0.}

\subsection{Contraintes technologiques}
\paragraph{}
Tacos est un système d'exploitation écrit dans le langage C. Il vient donc avec des librairies écrites dans ce même langage, il est donc naturel à ce que nous choisissions ce même langage ou un proche voisin pour développer son interface graphique.
\paragraph{}
Pour des raisons de conception et d'architecture logicielle, nous avons choisi le langage C++ pour écrire notre interface graphique. . 
\paragraph{}
Le système d'exploitation ne dispose pas de compilateur mais utilise la même architecture matérielle que nos postes de développment. Ainsi nous pourrons compiler notre interface graphique avec le compilateur C/C++ installé sur nos machines.
\subsection{Contraintes temporelles}
\paragraph{}
Ce projet étant un cas d'école le temps qui y est consacré est fortement lié à la charge de cours qui nous est imposée. 
De plus les vacances scolaires et les examens peuvent accentuer le manque de disponibilité.
\paragraph{}
Nous disposons d'à peu près deux heures de projet tutoré par semaine. Ce temps est normalement consacré à la réalisation. Nous travaillons de plus chez nous le soir quand la charge de travail nous le permet.
\subsection{Gestion des risques}
Toutes les taches effectuées dans ce projet peuvent être évaluées d'un point de leur difficulté des risques risques encourus.
\subsubsection{Tableau de gestion des risques}
\begin{tabular}{|p{1cm}|p{4cm}|p{1.5cm}|c|p{4cm}|}
  \hline
  Type & Description & Gravité & Statut & Solution \\
  \hline
  Ext & Dépendance du projet père Tacos & Bloquant & \color{green} \begin{picture}(10, 10) \put(0, 0){\circle*{10}} \end{picture} & Création d'un émulateur TacOS pour un environnement Linux \\
  \hline
  Ext & Portage des fonctionnalités de l'émulateur sur TacOS & Bloquant & \color{orange} \begin{picture}(10, 10) \put(0, 0){\circle*{10}} \end{picture} & Création d'une librairie partagée libTacOS surchargeant certaines fonctions de la libC. Utilisation du \textit{device file system} afin d'accéder aux périphériques \textbf{souris} et \textbf{clavier}. Portage de la librairie partagée et des drivers manquant sur TacOS le jour de la transition \\
  \hline
  Ext & Portage du système de fenêtrage sur TacOS & Bloquant & \color{red} \begin{picture}(10, 10) \put(0, 0){\circle*{10}} \end{picture} & Travail à faire \\  
  \hline
  Int & Apprentissage des systèmes de fenêtrages, gestionnaires de fenêtres et librairies de widget existantes & Bloquant & \color{orange} \begin{picture}(10, 10) \put(0, 0){\circle*{10}} \end{picture} & Au cas par cas, nous étudions les solutions existantes afin de retrouver des bonnes pratiques ou afin d'éviter les erreurs de conception connues \\
  \hline
\end{tabular}
\paragraph{Légende} 
\begin{description}
\item[Ext] Externe
\item[Int] Interne
\end{description}
%revues