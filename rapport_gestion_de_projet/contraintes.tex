\subsection{Contraintes technologiques}
\paragraph{}
Tacos est un système d'exploitation écrit dans le langage C. Il vient donc avec des librairies écrites dans ce même langage, il est donc naturel à ce que nous choisissions ce même langage ou un proche voisin pour développer son interface graphique.
\paragraph{}
Pour des raisons de conception et d'architecture logicielle, nous avons choisi le langage C++ pour écrire notre interface graphique. La librairie \textit{pron} implémentant notre protocole \textbf{client/serveur} est quand à elle écrite en C. 
\paragraph{}
Le système d'exploitation ne dispose pas de compilateur mais utilise la même architecture matérielle que nos postes de développment. Ainsi nous pourrons compiler notre interface graphique avec le compilateur C/C++ installé sur nos machines.
\subsection{Contraintes temporelles}
\paragraph{}
Ce projet étant un cas d'école le temps qui y est consacré et fortement lié à la charge de cours qui nous est imposée. 
De plus les vacances scolaires et les examens peuvent accentuer le manque de disponibilité.
\paragraph{}
Nous disposons d'à peu près deux heures de projet tutoré par semaine. Ce temps est normalement consacré à la réalisation. Nous travaillons de plus chez nous le soir quand la charge de travail nous le permet.
\subsection{Gestion des risques}
\subsubsection{Tableau de gestion des risques}