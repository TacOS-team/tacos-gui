\definecolor{red}{rgb}{1., 0., 0.}
\definecolor{green}{rgb}{0., 1., 0.}
\definecolor{orange}{rgb}{1., 0.39, 0.}

\subsection{Contraintes technologiques}

TacOS est un système d'exploitation écrit en langage C. Il vient avec des librairies écrites dans ce même langage, il est donc naturel que nous choisissions ce langage ou un langage voisin pour développer son interface graphique.\\

Pour des raisons de conception et d'architecture logicielle, nous avons choisi le langage C++ pour développer notre projet. Le système d'exploitation ne dispose pas de son propre compilateur, mais nous pouvons cross-compiler notre interface graphique avec le compilateur C/C++ installé sur nos postes de développement.

\subsection{Contraintes temporelles}

La réalisation de notre projet demande beaucoup d'investissement personnel, notamment en termes de temps. Évidemment, le temps dont nous disposons pour nous consacrer au projet est fonction de la charge de travail qui nous est imposée par nos études, et nous ne pouvons pas nous permettre de le laisser monopoliser tout notre temps libre. Ainsi, la vitesse d'avancement du projet subit quelques variations, notamment pendant les vacances scolaires ou les périodes d'examens.\\

Chaque semaine, un créneau d'environ deux heures est réservé dans notre emploi du temps afin que l'on se consacre au projet tutoré. Nous exploitons ces créneaux autant qu'il nous est possible de le faire mais en réalité, la majorité du développement se fait sur notre temps libre et nous travaillons bien plus que deux heures par semaine.

\subsection{Gestion des risques}
  Les aspects principaux du projet peuvent être évalués selon leur niveau de criticité. Nous avons ainsi défini une échelle pour les classifier : risques conceptuels, risques méthodologiques, risques bloquants et risques inconnus.

\begin{table}[h]
\centering
\begin{tabular}{|p{1.5cm}|p{4cm}|p{2.5cm}|c|p{5cm}|}
  \hline
  Type & Description & Gravité & Statut & Solution \\
  \hline
  Externe & Dépendance du projet père TacOS & Méthodologique & \color{green} \begin{picture}(10, 10) \put(0, 0){\circle*{10}} \end{picture} & Création d'un émulateur TacOS pour un environnement Linux. \\
  \hline
  Externe & Portage des fonctionnalités de l'émulateur sur TacOS & Bloquant & \color{orange} \begin{picture}(10, 10) \put(0, 0){\circle*{10}} \end{picture} & Création d'une librairie partagée \textit{libtacos} surchargeant certaines fonctions de la libC. Utilisation du \textit{devfs}\footnotemark[1] afin d'accéder aux périphériques \textbf{souris} et \textbf{clavier}. Portage de la librairie partagée et des drivers manquant sur TacOS en fin de projet (version 1.0). \\
  \hline
  Externe & Portage du système de fenêtrage sur TacOS & Bloquant & \color{red} \begin{picture}(10, 10) \put(0, 0){\circle*{10}} \end{picture} & Travail à faire. \\  
  \hline
  Interne & Documentation sur les systèmes de fenêtrages, gestionnaires de fenêtres et librairies de widgets existantes & Conceptuel & \color{orange} \begin{picture}(10, 10) \put(0, 0){\circle*{10}} \end{picture} & Au cas par cas, nous étudions les solutions existantes afin de retrouver des bonnes pratiques et d'éviter les erreurs de conceptions connues. \\
  \hline
\end{tabular}
\caption{Tableau de gestion des risques}
\end{table}

\footnotetext[1]{Device File System}
