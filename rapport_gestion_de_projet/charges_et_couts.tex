\subsection{Estimation des coûts}

Ce tableau présente notre estimation du coût horaire estimé pour chaque lot (``version'') de notre projet. Le découpage en versions est détaillé en section \ref{chrono}.

\begin{table}[h]
\centering
\begin{tabular}{|l|c|}
  \hline
  \rowcolor{dark_grey} Lot & Estimation (en heures) \\
  \hline
  Émulateur et bases du système de fenêtrage & 30 \\
  \hline
  Gestion des événements et gestionnaire de fenêtres basique & 40 \\
  \hline
  Système de fenêtrage complet & 30 \\
  \hline
  Gestionnaire de fenêtres avancé & 30 \\
  \hline
  Librairie de widgets  & 70 \\
  \hline
  Intégration avec TacOS & 20 \\
  \hline
  Développement de programmes de démo & 50 \\
  \hline
  \rowcolor{dark_grey} Total & 270 \\
  \hline
\end{tabular}
\caption{Estimation des coûts associés à chaque lot, en heures}
\end{table}

\subsection{Estimation des ressources}

Nous prévoyons d'allouer environ 500 heures au total, réparties sur les 4 membres du projet. Le tableau suivant présente notre estimation de la capacité de production des membres de l'équipe, en heures, pour chaque semaine du projet.

\begin{table}[h]
\centering
\begin{tabular}{|l|c|c|c|c|c|c|c|c|c|c|c|c|c|}
  \hline
  \rowcolor{dark_grey} Semaine           &   6   &   7   &   8   &   9     &   10  &   11  &   12  &   13    &   14  &   15    &   16    &   17    &  Total \\
  \hline
  Nicolas Hug       &   8   &   9   &   8   &   14    &   9   &   9   &   9   &   12    &   9   &   10    &   13    &   13    &   123 \\
  \hline
  Julien Marchand   &   9   &   9   &   11  &   11    &   11  &   4   &   12  &   12    &   11  &   13    &   14    &   14    &   131 \\
  \hline
  Benoit Morgan     &   9   &   8   &   8   &   10    &   12  &   12  &   6   &   6     &   8   &   15    &   15    &   15    &   124 \\
  \hline
  Adrien Sulpice    &   10  &   10  &   9   &   10    &   10  &   12  &   11  &   11    &   12  &   12    &   12    &   13    &   132 \\
  \hline
  \rowcolor{dark_grey} Total             &   36  &   36  &   36  &   45    &   42  &   37  &   38  &   41    &   40  &   50    &   54    &   55    &   510  \\
  \hline
\end{tabular}
\caption{Estimation de la capacité de production de chaque membre de l'équipe, en heures}
\end{table}

\vspace{1em}

Cela peut sembler important au regard de la charge horaire estimée (\~270h), mais nous prévoyons volontairement une charge de travail plus conséquente car :
\begin{itemize}
\renewcommand{\labelitemi}{$\bullet$}
\item Le temps que nous devrons consacrer à la gestion du projet, (planification/affectation des tâches, rédaction des compte-rendus d'avancement, réunions avec les tuteurs...) n'est pas compris dans les estimations horaires de chaque lot.
\item Certaines tâches (de conception notamment, mais possiblement d'implémentation et de correction de bugs) nécessiteront une synchronisation de l'équipe et nous perdrons alors la parralélisation des tâches (si nous sommes 3 ou 4 sur la même tâche, cela consommera 3 ou 4 fois plus de temps).
\item Nous sommes conscients que nous aurons forcément des aléas qui nous obligeront à dévier de notre planning et à consommer davantage de temps ! 
\end{itemize}

\subsection{Tableau de compétences des acteurs}

Ce tableau présente les niveaux de compétence des membres de l'équipe dans chacun des domaines en rapport avec le projet (technologies, outils, culture informatique...).

\begin{table}[h]
\centering
\begin{tabular}{|l|c|c|c|c|c|c|c|c|}
  \rowcolor{dark_grey} \hline &  C \footnotemark[1] & C++ \footnotemark[2] & TacOS & Git & Redmine & WS \footnotemark[3] & WM \footnotemark[4] & Widgets \footnotemark[5] \\
  %                 &   C     &   C++   & Tacos &   Git   & Redmine &   WS    &  WM     & Widgets
  \hline
  Nicolas Hug       &   ++    &   ++    &   +   &   ++    &   ++    &   +     &   ++    &   +++ \\
  \hline
  Julien Marchand   &   +++   &   ++    &  +++  &   ++    &   +++   &   +++   &   ++    &   + \\
  \hline
  Benoit Morgan     &   +++   &   ++    &   +   &   ++    &   +++   &   ++    &   ++    &   + \\
  \hline
  Adrien Sulpice    &   ++    &   +++   &   ++  &   ++    &   +++   &   +     &   +++   &   + \\
  \hline
\end{tabular}
\caption{Niveaux de compétences des membres du projet}
\end{table}

\footnotetext[1]{low-level}
\footnotetext[2]{high-level}
\footnotetext[3]{Fonctionnement des systèmes de fenêtrage, i.e. X Window.}
\footnotetext[4]{Fonctionnement des gestionnaires de fenêtres, i.e. Metacity, Awesome...} 
\footnotetext[5]{Fonctionnement des librairies de widgets, i.e. Xaw, Motif, GTK+, Qt...} 

\subsection{Tableau des compétences nécessaires aux lots}

Ce tableau présente les compétences nécessaires à la réalisation des différents lots. Couplé au tableau de compétences des acteurs, il nous a aidé à affecter les différentes tâches (présentées section \ref{taches}) aux différents membres du projet.

\begin{table}[h]
\centering
\begin{tabular}{|l|c|c|c|c|c|c|c|c|}
  \rowcolor{dark_grey} \hline &  C & C++ & TacOS & Git & Redmine & WS & WM & Widgets \\
  %                 &   C     &   C++   & Tacos &   Git   & Redmine &   WS    &  WM     & Widgets
  \hline
  Émulateur/bases du système de fenêtrage & \cellcolor[gray]{0} & \cellcolor[gray]{0.5} & \cellcolor[gray]{0} & \cellcolor[gray]{0.9} & \cellcolor[gray]{0.9} & \cellcolor[gray]{0} & \ \cellcolor[gray]{1} & \cellcolor[gray]{1} \\
  \hline
  Événements/WM basique & \cellcolor[gray]{0.75} & \cellcolor[gray]{0.25} & \cellcolor[gray]{1} & \cellcolor[gray]{0.9} & \cellcolor[gray]{0.9} & \cellcolor[gray]{0.25} & \ \cellcolor[gray]{0.5} & \cellcolor[gray]{1} \\
  \hline
  Système de fenêtrage complet & \cellcolor[gray]{0.5} & \cellcolor[gray]{0.25} & \cellcolor[gray]{1} & \cellcolor[gray]{0.9} & \cellcolor[gray]{0.9} & \cellcolor[gray]{0} & \ \cellcolor[gray]{0.75} & \cellcolor[gray]{1} \\
  \hline
  Gestionnaire de fenêtres avancé & \cellcolor[gray]{1} & \cellcolor[gray]{0} & \cellcolor[gray]{1} & \cellcolor[gray]{0.9} & \cellcolor[gray]{0.9} & \cellcolor[gray]{0.75} & \ \cellcolor[gray]{0} & \cellcolor[gray]{1} \\
  \hline
  Librairie de widgets & \cellcolor[gray]{1} & \cellcolor[gray]{0} & \cellcolor[gray]{1} & \cellcolor[gray]{0.9} & \cellcolor[gray]{0.9} & \cellcolor[gray]{1} & \ \cellcolor[gray]{1} & \cellcolor[gray]{0} \\
  \hline
  Intégration avec TacOS & \cellcolor[gray]{0} & \cellcolor[gray]{1} & \cellcolor[gray]{0} & \cellcolor[gray]{0.9} & \cellcolor[gray]{0.9} & \cellcolor[gray]{1} & \ \cellcolor[gray]{1} & \cellcolor[gray]{1} \\
  \hline
  Développement de programmes de démo & \cellcolor[gray]{1} & \cellcolor[gray]{0.5} & \cellcolor[gray]{1} & \cellcolor[gray]{0.9} & \cellcolor[gray]{0.9} & \cellcolor[gray]{1} & \ \cellcolor[gray]{1} & \cellcolor[gray]{0.25} \\
  \hline
\end{tabular}
\caption{Compétences nécessaires à la réalisation de chaque lot}
\end{table}
