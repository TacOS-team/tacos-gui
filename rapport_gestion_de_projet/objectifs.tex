Pour réaliser ce projet, nous avons définit les objectifs suivants. Le numéro préfixé détermine les tâches qui peuvent être paralélisées. On peut faire les tâches de même niveau en même temps mais on ne peut pas commencer les étapes de niveau trois (par exemple) si les étapes de niveau un et deux n'ont pas été terminées.

\begin{itemize}
\renewcommand{\labelitemi}{$\bullet$}
\item 1 - Création d'un émulateur pour linux
\item 2 - Socket TacOS domain
\item 2 - Définition du protocole de communication Client/Serveur X like
\item 3 - Implémentation du système Client/Serveur
\item 3 - Spécification du gestionnaire de fenêtres flottant
\item 4 - Implémentation du gestionnaire de fenêtres flottant
\item 4 - Définition de l'api de widgets
\item 4 - Implémentation de l'api de widgets
\item 5 - Création d'applications telles que un gestionnaire de fichiers, un editeur de texte et un émulateur de terminal.
\end{itemize}
