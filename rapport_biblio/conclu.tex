Cette étude bilbiographique nous a permis d'acquérir une vision globale des interfaces graphiques, tout en mettant l'accent sur les points spécifiques qui nous seront utiles pour implémenter notre solution. Nous avons, en particulier, pu discerner bien distinctement les différents acteurs d'une interface graphique, dont les frontières sont habituellement assez floues.

Nous avons été doublement séduits par le modèle X Window : tout d'abord, l'architecture client/serveur est intéressante puisqu'elle permet une utilisation très souple du gestionnaire de fenêtrage. Ensuite, la dissociation des différentes couches logicielles, chacune remplissant un rôle bien défini, permet de conserver une certaine simplicité, et confère à l'ensemble une grande modularité. 

Nous allons donc nous inspirer de X Window pour notre système de fenêtrage, et implémenter une architecture client/serveur, même si nous ne disposons pas du réseau à l'heure actuelle. Nous implémenterons ensuite un gestionnaire de fenêtres flottant simple, puisque c'est cette solution qui nous semble être la plus ergonomique. Enfin, nous développerons une librairie de wigets qui permettra la création d'applications graphiques que nous utiliserons pour tester et mettre en oeuvre notre solution : une horloge, un éditeur de texte, ou encore des petits jeux de type casse brique.

