%partie présentation, j'ai déjà créé le fichier pour inclure l'image du slip

\author{Benoît Morgan \texttt{<morgan@etud.insa-toulouse.fr>}} % le tuning
\subsection{L'équipe}
\frame{ 
  \frametitle{Les membres}
  \begin{figure}[H]
    \centering
      \includegraphics[width=9cm]{figures/equipe}
  \end{figure}
}

\frame{ 
  \frametitle{Les encadrants}
  \small
  \begin{description}
    \item[Pierre-Emmanuel Hladik] enseignant-chercheur au LAAS-CNRS
    \item[Maxime Chéramy] doctorant au LAAS-CNRS
  \end{description}
  \normalsize
}

\author{Adrien Sulpice \texttt{<sulpice@etud.insa-toulouse.fr>}} % pas le tuning
\subsection{Enjeux et objectifs}
  \frame{
    \frametitle{Enjeux}

    \begin{itemize}
      \item Un système de fenêtrage que nous avons appelé “pron”
      \item Un gestionnaire de fenêtres flottant, “guacamole”
      \item Une librairie de widgets, “sombrero”
      \item Un environnement de bureau minimaliste, “salsa”
    \end{itemize}
  }

  \frame{
    \frametitle{Objectifs}

    \footnotesize
    \begin{itemize}
      \item \circled{1} Création d’un émulateur pour Linux
      \item \circled{2} Sockets TacOS domain
      \item \circled{2} Définition du protocole de communication client/serveur X-like
      \item \circled{3} Implémentation du système de fenêtrage client/serveur
      \item \circled{3} Spécification du gestionnaire de fenêtres flottant
      \item \circled{4} Implémentation du gestionnaire de fenêtres flottant
      \item \circled{4} Définition de la librairie de widgets
      \item \circled{4} Implémentation de la librairie de widgets
      \item \circled{5} Création d’applications pour l’environnement de bureau
    \end{itemize}
    \normalsize
  }

\subsection{Démonstration}
  \frame{
    \frametitle{Démonstration}
    \begin{figure}[H]
      \centering
      \includegraphics[width=225px]{figures/image_demo}
    \end{figure}
  }
