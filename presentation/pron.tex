\subsection{pron}
\frame{
  \frametitle{pron}

  \begin{block}{Rôle et fonctionnalités}
    \begin{itemize}
      \item Système de fenêtrage ("serveur graphique")
      \item Permet aux clients (applications) de créer des fenêtres (zones de dessin rectangulaires)
      \item Propose des primitives de dessin basiques (point, ligne, rectangle, ellipse, texte...)
      \item Dialogue avec les périphériques
        \begin{itemize}
          \item entrée : clavier, souris (gestion des événements)
          \item sortie : écran
        \end{itemize}
    \end{itemize}
  \end{block}
}

\subsection{pron}
\frame{
  \frametitle{pron}

  \begin{block}{Architecture}
    \begin{itemize}
      \item Screen
      \item Drawable
      \item GC
      \item Client
      \item Keyboard/Mouse
    \end{itemize}
  \end{block}
}

\subsection{pron}
\frame{
  \frametitle{pron}

  \begin{block}{Drawable}
    \begin{itemize}
      \item Entité dans laquelle on peut dessiner
      \item Toutes les méthodes de dessin sont polymorphes et acceptent des Drawable
      \item Deux types :
        \begin{itemize}
          \item Window : fenêtre affichée à l'écran. Elle n'est pas stockée en mémoire : si elle est obscurcie, l'application qui l'a créée doit la redessiner (entièrement ou en partie).
          \item Pixmap : zone de dessin stockée en mémoire. Elle peut être copiée dans une Window à tout moment.
        \end{itemize}
    \end{itemize}
  \end{block}
}

\subsection{pron}
\frame{
  \frametitle{pron}

  \begin{block}{Window}
    \begin{itemize}
      \item Fenêtre (zone de dessin rectangulaire) affichée à l'écran.
      \item Forment un arbre n-aire, l'ordre des fils d'une fenêtre détermine leur profondeur à l'intérieur de cette fenêtre
      \item Une fenêtre ne peut pas dessiner :
        \begin{itemize}
          \item En dehors de son père
          \item Dans une zone occupée par une soeur de pronfondeur inférieure
        \end{itemize}
      \item Gestion d'une zone de clipping
    \end{itemize}
  \end{block}
}

\subsection{pron}
\frame{
  \frametitle{pron}

  \begin{block}{GC (Graphical context)}
    \begin{itemize}
      \item Contexte graphique
      \item Contient les paramètres de dessin à un instant donné :
        \begin{itemize}
          \item Couleur d'avant-plan
          \item Couleur d'arrière-plan
          \item Police de caractères utilisée
          \item ...
        \end{itemize}
      \item Les primitives de dessin utilisent le contexte graphique courant, préparé avant leur appel
    \end{itemize}
  \end{block}
}

\subsection{pron}
\frame{
  \frametitle{pronlib}

  \begin{block}{}
    \begin{itemize}
      \item Librairie cliente de pron
      \item Permet aux applications d'interagir avec le serveur graphique :
      \begin{itemize}
        \item Connexion
        \item Création
      \end{itemize}
      \item Librairie relativement bas niveau
        \begin{itemize}
          \item Une primitive de la pronlib correspond grosso-modo à un appel de méthode de pron 
          \item Rien d'"intelligent"
        \end{itemize}
    \end{itemize}
  \end{block}
}
