\section{Introduction}

Ce document est une documentation technique du projet TacOS-GUI.
Il est à lire après notre état de l'art sur les interfaces graphiques.
Nous présenterons l'architecture générale du projet, jusqu'à détailler le fonctionnement interne des différents composants.
Vous pourrez trouver quelques exemples de code, mais ce n'est pas l'objectif de ce document :
si vous souhaitez une documentation exhaustive du code source, référez-vous à la documentation Doxygen, qui est très complète.
Le code source est lui-même assez bien commenté.

Dans un premier temps, nous présenterons l'architecture générale du projet, les choix de conception que nous avons faits et nous expliquerons la méthodologie de travail que nous avons adoptée pour mener à bien ce projet.
Ensuite, nous détaillerons les différents acteurs de notre projet et leur fonctionnement : le système de fenêtrage Pron, le gestionnaire de fenêtres Guacamole et la librairie de widgets Sombrero.
Enfin, nous ferons un bilan de ce que ce projet a apporté au système d'exploitation TacOS.
