\section{Apports à TacOS}

Malgrès notre investissement, nous n'avons pas eu le temps de porter ce projet sur TacOS dans sa version finale.
En effet, le projet dépends de beaucoup de fonctions qui ne sont pas implémentées sous TacOS.
Notamment la STL
\footnote{Librairie standard du C++ contenant vector, map, list, etc.}.
Il a donc été nécessaire d'implémenter ces classes pas nous-même.
Lors de la tentative de portage finale, nous avons remarqué qu'il en manquait encore quelques-unes.
Notamment la liste chainée, nous n'avions implémenté que la liste doublement chaînée.

Pour ce projet, nous avons décidé d'utiliser le C++.
Le C++ était disponible par défaut sous TacOS sauf pour :

\begin{itemize}
  \item La fonction new : il a été nécessaire de l'implémenter en utilisant malloc()
  \item Les fonctions delete : pareil en utilisant free()
  \item Les exceptions : pour gérer les exceptions, il est nécessaire d'utiliser plusieurs librairies demandant beaucoup de dépendances. Nous n'avons pas utilisé les exceptions pour cette raison
\end{itemize}

Voici une liste non exhaustive des apports de TacOS-GUI à TacOS :
  
\begin{itemize}
  \item STL (C++)
  \item libc
    \begin{itemize}
      \item qsort
      \item gettimeofday
    \end{itemize}
  \item driver
    \begin{itemize}
      \item vesa
      \item vga
      \item souris
    \end{itemize}
\end{itemize}

Une fois que le portage aura été effectué, ce projet apportera plusieurs logiciels graphiques à TacOS.
