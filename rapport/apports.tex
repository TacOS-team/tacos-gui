\section{Apports à TacOS}

Comme précisé dans la partie méthodologie, la librairie libtacos, couche d'abstraction de TacOS, pour le développement de TacOS-GUI doit être implémentée sous TacOS.

Pour ce projet, nous avons décidé d'utiliser le C++.
Le C++ était disponible par défaut sous TacOS sauf pour :

\begin{itemize}
  \item La fonction new : il a été nécessaire de l'implémenter en utilisant malloc()
  \item Les fonctions delete : pareil en utilisant free()
  \item Les exceptions : pour gérer les exceptions, il est nécessaire d'utiliser plusieurs librairies demandant beaucoup de dépendances. Nous n'avons pas utilisé les exceptions pour cette raison
\end{itemize}

Voici une liste non exhaustive des apports de TacOS-GUI à TacOS :
  
\begin{itemize}
  \item STL (C++)
  \item libc
    \begin{itemize}
      \item qsort
      \item gettimeofday
    \end{itemize}
  \item driver
    \begin{itemize}
      \item vesa
      \item vga
      \item souris
    \end{itemize}
\end{itemize}

