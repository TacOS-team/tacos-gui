Il y a deux ans, dans le cadre des projets tutorés, un système d'exploitation 32 bits a été développé par des étudiants : TacOS.
Nous souhaitions à notre tour contribuer à ce projet, et le développement d'une interface graphique nous a semblé intéressant.
En effet, cet axe de développement permet d'adopter à la fois une approche bas niveau (conception de drivers, gestion des mecanismes propres au système d'exploitation), et une approche haut niveau (gestion des fenêtres, élaboration d'une API, etc.).

Cette étude bibliographique nous permettra de découvrir la structure et le fonctionnement de quelques interfaces graphiques existantes, afin de concevoir notre propre solution adaptée à TacOS.
Dans une première partie, nous présenterons diverses généralités sur les interfaces graphiques, puis dans la seconde nous détaillerons l'architecture des systèmes existants, en mettant l'accent sur la solution libre la plus répandue : X Window System.
