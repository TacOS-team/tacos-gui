\section{Sombrero}

Sombrero est la librairie de widgets de TacOS-GUI.
Son rôle est d'apporter au programmeur un niveau d'abstraction supplémentaire par rapport à la Pronlib, afin de faciliter la création d'interfaces graphiques pour les applications.

\subsection{Conception de la librairie}

La plupart des librairies de Widgets veulent être multi-plateformes.
Pour cela, elles utilisent au minimum le système de fenêtrage et gèrent en interne un maximum de choses.
Le développement de Sombrero ayant un but pédagogique, nous avons au contraire essayer d'exploiter au maximum les fonctionnalités de Pron et de faire le mimimum dans la librairie.
Dans cette optique, nous avons décidé que à tout widget correspondait une Window dans pron.
Ainsi c'est pron qui détermine à qui transmettre les évènements claviers, souris, expose, etc.

\subsubsection{Widgets}

Nous avons distingué deux grandes familles d'éléments graphiques :

\begin{itemize}
  \item Les widgets : ce sont tous les éléments graphiques de l'interface : boutons, labels, textarea, etc.
  \item Les conteneurs : ils contiennent des widgets (ou d'autres conteneurs) et permettent de les disposer dans la fenêtre.
\end{itemize}

Tout objet susceptible de contenir d'autres widgets héritera donc de container.
Pour gérer facilement ces imbrications, un conteneur est lui-même un Widget et a sa propre Pron Window.
On retrouvera bien chez pron un arbre de fenêtres correspondant à la hiérarchie des widgets.

Voici le diagramme de classes de la librairie. On peut voir les différentes classes conteneurs et les différentes classes de Widget implémentées :

\begin{figure}[H]
  \centering
  \includegraphics[width=17.5cm]{images/classDiagWidget.png}
  \caption{Diagramme de classes de Sombrero}
  \label{fig:diagrame_classes_sombrero}
\end{figure}

La classe Widget est bien à la racine.
Elle contient tout ce qui est commun à tous les widgets dont notamment la position.
Il est à noter que, contrairement à Pron, la position est relative au widget parent.
Ceci facilite le placement des widgets par les conteneurs, ils sont placés relativement au conteneur et pas en position absolue.

Dans les conteneurs, on peut voir la classe Bin.
Cette classe est la classe mère de tous les conteneurs ne contenant qu'un seul élément.
Notamment les scrollPanes et les Window.
La classe Window est une fenêtre qui sera décorée.
C'est le seul widget qui peut être décoré.
Une Window peut ensuite contenir un seul widget.
Si on veut afficher une fenêtre décorée, c'est cette classe qu'il faut utiliser.
Ensuite il faut y mettre soit un conteneur soit un widget.

Prenons l'exemple de l'application gestionnaire de fichier.
Cette application contient un label pour afficher le répertoire courant ainsi que un bouton pour chaque fichier dans un scrollPane. On retrouve donc l'arborescence de widgets suivante :

\begin{center}
  \begin{tikzpicture}[
    grow via three points={one child at (0.5,-0.7) and
    two children at (0.5,-0.7) and (0.5,-1.4)},
    edge from parent path={(\tikzparentnode.south) |- (\tikzchildnode.west)}]
    \node [root] {Window}
      child { node {Conteneur Grid}
        child { node {Label}}
        child { node {ScrollPane}
          child { node {Conteneur Grid}
            child { node {Bouton Fichier 1}}
            child { node {Bouton Fichier 2}}
            child { node {Bouton Fichier 3}}
            child { node {...}}
          }
        }
      };
  \end{tikzpicture}
\end{center}

\subsubsection{Gestion de la Pron Window}

Instinctivement, on voudrait créer la pronWindow au moment de la construction de Widget.
Il faudrait donc le faire dans le constructeur.
Ceci marche très bien pour un un objet fenêtre qui a pour PronWindow mère la root Window de pron.
Seulement pour les autres widgets imbriqués, dans le constructeur, on ne connait pas encore le widget père.
Si on crée une PronWindow avec comme fenêtre mère la root Window de Pron, on se retrouvera avec un widget décoré par Guacamole et ce n'est pas ce que l'on veut.
C'est pourquoi il faut ajouter une fonction setParent aux widgets.
Lorsqu'on définit le parent du widget avec setParent(), on peut donc créer la Pron window avec comme Window mère la Window du widget père.

Ceci fonctionne parfaitement mais cela oblige l'utilisateur à bien créer ses objets dans l'ordre et faire les setParent() dans l'ordre également.
Ceci est très contraignant.
Pour éviter cela, il suffit lors du setParent de regarder si le père a déjà été initialisé.
Si oui, on crée la Pron Window.
Si non, on ne fait rien.
Ceci nécessite donc aux classes de conteneurs de faire les setParents à tous ses widgets fils dans son propre setParent().
Ainsi, on peut créer toute notre arborescence de widgets et dès qu'on ajoute le premier widget dans la fenêtre, elle appelle la fonction setParent() de ce widget qui l'appelle à ses fils s'il en a et ainsi de suite.
L'utilisateur peut donc créer ses widgets dans n'importe quel ordre et faire le setParent quand il le veut ce qui est beaucoup plus agréable.

Pour les conteneurs, le code de setParent() étant le même pour tous, la fonction a été implémentée de la façon suivante dans la classe Container :

\lstinputlisting{listings/container_setparent.cpp}

La fonction getChildren() est une fonction virtuelle. Toutes classes dérivant directement ou indirectement de Container devra la surcharger et renvoyer tous ses fils dans un vector. Ainsi, plus aucun conteneur n'a à se soucier du setParent().

\subsubsection{Gestion des évènements}

Comme nous l'avons dit auparavant, nous avons laissé Pron gérer les évènements et les propager.
Chaque widget s'abonne aux évènements dont il a besoin.
Ces évènements sont ensuite traduits par des appels de fonctions virtuelles (handlers) de la classe Widget
\footnote{Nous verrons comment cela a été fait plus tard en parlant de l'application.}.
Par exemple lorsqu'un évènement souris arrive sur un widget, la fonction handleEventMouseButton() de celui-ci est appellée.
Lorsqu'on crée un nouveau widget, il suffit de surcharger les handlers qui nous intéressent. Il faut penser tout de même à rappeller la fonction handler de la classe parente.

En effet, nous avons mis dans les handlers des widgets un maximum de fonctionnalités communes au plus grand nombre des widgets.
Pour garder l'exemple des évènements des boutons de la souris, la fonction handleEventMouseButton() prend en paramètre un pointeur vers un pron::EventMouseButton.
Cet objet contient l'état des boutons et pas quel bouton a changé.
Pour détecter un clic, il faut donc sauvegarder l'état des boutons et regarder à chaque nouvel évènement lequel a changé.
Pour éviter de le coder dans chaque widget, cette fonction a été implémentée dans la classe Widget comme ça le code n'est pas dupliqué.
Dans cette fonction, on détecte les clics sur les différents boutons, double clic, etc.
Dès qu'un évènement est détecté, on appelle une nouvelle méthode virtuelle spécifique à l'évènement.
Voici le code de cette fonction. On remarque que avec le clic gauche, il y a également le clic et double clic à gérer en plus des mouseDown et mouseRelease :

\lstinputlisting{listings/handleEventMouseButton.cpp}

Si on crée maintenant un widget bouton, il suffit de surcharger la méthode handleClick() et de mettre le code correspondant à cet évènement à l'intérieur.
Pas besoin de coder de nouveau la gestion des boutons.

\subsubsection{Mise à jour des widgets : update()}

Lors de la gestion d'un évènement, il est courant de devoir mettre à jour certaines positions ou modifier l'agencement des widgets ou tout autre mise à jour de l'interface.
Lors de ces mises à jour, on modifie les variables dans widgets.
Seulement cela ne modifie pas ces informations dans Pron.
Il est donc nécessaire d'envoyer un message à Pron pour le notifier des changements.
Pour cela, il y a trois fonction à appeller :

\begin{itemize}
  \item updatePronSize : cette fonction met la taille à jour si elle a changé
  \item updatePronPosition : met à jour la position
  \item updateBGColor : met à jour la couleur de fond
\end{itemize}

Après un changement d'un seul paramètre, on peut n'appeller qu'une seule de ces fonctions.
Lors d'une mise à jour, un conteneur va modifier les informations des widgets qu'il contient.
Sachant que ces widgets peuvent être eux-même des conteneurs, si on a modifié leurs dimensions, ils vont devoir de nouveau mettre à jour leur propres fils et ainsi de suite jusqu'aux feuilles de l'arbre.
Pour généraliser cette mise à jour, il y a une fonction update() commune à tous les widgets.
Par défaut elle ne fait qu'appeler les trois fonctions de mise à jour citées précédemment.
Si le widget a besoin de faire des choses supplémentaires
\footnote{Par exemple le conteneur de grille va redimensionner tous ses widgets fils si on modifie sa taille.}
lors d'un update, il suffit de surcharger la fonction update().

Cette méthode fonctionne bien mais il s'avère que bien souvent les mises à jours bouclent.
En effet, il est fréquent de mettre à jour un widget, de notifier Pron de cette mise à jour, qui lui-même renvoie un nouvel évènement au widget qui relance une mise à jour.
Pour éviter ce problème, nous avons créé un code commun de update() à tous les widgets et déplacé le code de mise à jour dans la fonction execUpdate().
Si on veut modifier la mise à jour il faut surcharger la fonction execUpdate() et surtout pas la fonction update().
La fonction update() commune va unmaper la fenêtre, la mettre à jour puis la remapper. Ceci limite considérablement le nombre d'évènements et décharge Pron.
Pour éviter des boucles dans la librairie elle-même, on sauvegarde également si l'objet est en train d'être mis à jour pour éviter qu'un de ses fils le remmette à jour et ainsi créer une boucle.
Le code de la fonction d'update est donc le suivant :

\lstinputlisting{listings/widget_update.cpp}

Attention : dans les surcharges des fonctions execUpdate(), il est nécessaire d'appeler la fonction de la classe parente sinon l'objet ne sera pas mis à jour dans Pron.

\subsubsection{Dessin des widgets}

Pour dessiner un widget, il suffit qu'il dessine dans sa propre Pron Window.
Le code du dessin doit tout simplement être mis dans la fonction draw().
Cette fonction sera automatiquement appellée à chaque réception d'un évènement d'exposition de Pron
\footnote{Il n'est pas nécessaire de l'appeler après un update, ce sera également fait automatiquement par Pron après le remap qui enverra un évènement d'exposition.}.

Remarque : Lors de la rédaction de ce document, cette fonction redessine l'intégralité du widget.
Lors de la réception d'un évènement d'exposition, Pron précise quelle a été la nouvelle zone exposée.
Une amélioration serait donc de ne redessiner qu'à l'intérieur de cette zone et pas redessiner systèmatiquement tout le widget.
Cependant cela n'est pas toujours facile à faire.

\subsection{La classe Application}

La classe Application est l'objet permettant de communiquer avec Pron.
C'est elle qui englobe la boucle d'évènements.
C'est une classe singleton.
Si le constructeur est appelé deux fois, l'application est tout de suite fermée avec un message d'erreur.
L'application tient à jour une map contenant la correspondance entre un identifiant de Pron Window et un widget.
Tous les widgets ont accès à l'instance de l'application.
Lors de la création de leur Pron Window, ils se référencent automatiquement auprès de la classe application.
A la fin du main, lorsque l'utilisateur a créé tous ses widgets, il appelle la fonction sombrerun() de la classe Application.
Cette fonction exécute la boucle d'évènements.
A chaque fois qu'elle récupère un évènement, elle trouve le widget correspondant et appelle les fonctions handle que nous avons présentées auparavant.

Ceci marche parfaitement pour les évènements sur les widgets.
Seulement parfois il est nécessaire de s'abonner à d'autres évènements tels que les créations de fenêtres, etc.
Ceci n'est donc pas spécifique à un widget.
Pour gérer ces cas, dans la fonction sombrerun(), avant d'essayer de transmettre l'évènement à un widget, on regarde si c'est un des évènements ``général''.
Si oui, on appelle une fonction handler virtuelle.
Pour accéder à ces évènements, il suffit donc de surcharger la classe Application et d'implémenter ces fonctions.

\lstinputlisting{listings/sombrerun.cpp}

Attention : L'application étant une instance unique, si vous voulez utiliser votre instance d'Application, il faut qu'elle soit instanciée avant le premier Application::getInstance() soit avant la création d'un widget.
Sinon ce sera l'application par défaut qui sera appelée.

\subsection{Signaux et slots}

Nous avons vu que pour gérer les évènements, il y avait différents handlers à surcharger si besoin est.
Ceci est très pratique pour le développement de la bibliothèque mais pour l'utilisateur final ce n'est pas très pratique.
Pour résoudre ce problème, nous avons décidé de nous inspirer du système de signaux de Qt
\footnote{Qt est une librairie multi-plateforme développée par Nokia. Plus d'informations sur le site officiel : http://qt.nokia.com/products/}.

Ce système correspond au design pattern observeur/observable.
Nous avons utilisé une librairie qui proposait une implémentation
\footnote{La librairie utilisée est Sigslot. Nous allons bientôt utiliser notre propre implémentation car il lui manque quelque fonctionnalités comme l'éméteur du signal.}.
Les signaux peuvent être émis par différentes classes.
Il est possible d'envoyer des variables en paramètres de ces signaux.
A un signal, il est possible de connecter des slots qui sont en fait de simples fonctions d'une classe ayant les mêmes paramètres que le signal.
La classe contenant les slots doit dériver de la classe has\_slots.

Le signaux sont en fait des objets des classes signal0 à signal8. Avec le nombre étant le nombre de paramètres du signal.
Ces classes sont génériques.
Elle prennent un paramètre template par paramètre du signal.

Voici un exemple simple de l'utilisation des signaux d'une scrollBar :

\lstinputlisting{listings/signalSlot.cpp}

On remarque que pour l'utilisateur c'est vraiment très pratique. Il n'a pas besoin de surcharger des handlers.
Il lui suffit de créer un slot et de le connecter.