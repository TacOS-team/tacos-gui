\section{Méthodologie}

\subsection{Enjeux}

Toute la difficulté du développement de TacOS-GUI réside dans le fait que ce projet soit intimement lié à un système d'exploitation en constante évolution.
Une des voies que nous aurions pu suivre est de développer directement pour TacOS, mais plusieurs problèmes découlent de ce choix.

Le premier est que TacOS, pour l'instant, n'implémente pas la totalité de la libC.
De plus, ayant choisi le C++ comme langage de programmation pour notre projet, nous aurions dû développer en premier lieu la STL sur TacOS.
Enfin, TacOS-GUI utilisant un système de fenêtrage client/serveur, il nous fallait un système de communication inter-processus (IPC) ``sockets-like'', ce qui impliquait le développement d'une librairie de sockets.

Tous ces prérequis auraient fortement retardé le début du développement de TacOS-GUI, et nous auraient empêché de paralléliser les tâches.
Nous avons donc opté pour la solution vivement conseillée par nos tuteurs : développer un émulateur qui nous permette de développer sous Linux, tout en simulant un environnement TacOS.

\subsection{Émulateur TacOS}

L'émulateur a été développé pour Linux.
Il permet en fait de coder Pron, Guacamole est les applications comme si elles étaient sous TacOS.
Pour cela, nous avons créé des librairies qui ont exactement les même fonctions que sous TacOS mais en interne elles ne font pas la même chose.
L'émulateur est juste une fenêtre créée par la SDL.
Lors des appels aux fonctions des drivers, au lieu de vraiment faire des appels systèmes, ce sont des appels à la SDL qui sont faits.
Au lieu de dessiner à l'écran, on dessine donc notre fenêtre simulant l'écran sans que l'application ait ``conscience'' qu'elle n'est pas sur TacOS.
Comme spécifié dans l'introduction de cette partie, voici les différents points que nous avons dû implémenter :

\begin{itemize}
  \item Environnement TacOS
  \item Drivers
  \begin{itemize}
    \item VESA
    \item Clavier
    \item Souris 
  \end{itemize}
  \item Sockets (\verb|libtsock|)
  \item LibC TacOS
  \begin{itemize}
    \item exec\_elf
  \end{itemize}
\end{itemize}

\subsubsection{Environnement TacOS}

L'environnement TacOS est décrit dans la librairie du même nom, \verb|libtacos|.

La libC de TacOS définissant un certain nombre de fonctions au nom identique à celles de la libC Linux, il nous a fallu surcharger ces fonctions pour leur associer le même comportement que sous TacOS.

Le cas le plus concret concerne les fonctions de manipulation des fichiers : open, read, write, close, ioctl...

Par exemple, sous Tacos, les drivers sont implémentés avec des devices.
Pour utiliser le driver VESA par exemple, il faut ouvrir le fichier /dev/vesa.
Il faut donc modifier la fonction open pour qu'elle simule ce comportement.
Voici donc le wrapper de la fonction open() :

\lstinputlisting{listings/open.c}

\subsubsection{Drivers}

L'émulation des différents drivers cités plus haut se fait par le biais d'une librairie graphique, la SDL, au dessus de X.org.
Le driver souris est implémenté grâce aux évènements souris de la SDL, les évènements clavier fonctionnent de la même manière.

Pour ce qui est du driver VESA, la SDL étant une librairie graphique, elle permet de créer un fenêtre X qui représentera notre écran physique. 

\subsubsection{Librairie de sockets}

Au moment du développement de la librairie de sockets \verb|libtsock|, il n'y avait pas d'IPC de ce type développée sous TacOS, nous avons donc pu développer la nôtre sans contrainte particulière, en commençant par une implémentation sur l'émulateur.

Cette première implémentation utilise les sockets UNIX domain. L'implémentation TacOS utilise des buffers alloués en espace noyau, qui représentent la file de messages de chaque socket.

Voici la liste exhaustive des primitives de la librairie de sockets que nous avons défini.

\lstinputlisting{listings/tsock.c}

\subsubsection{LibC TacOS}

En plus de la surcharge des fonctions de la libC Linux, nous avons dû définir les fonctions spécifiques à la libC TacOS. Exemple avec \verb|exec_elf| :
\lstinputlisting{listings/exec_elf.c}

\subsubsection{Travail en équipe}

Pour mener à bien ce projet, nous avons du mettre en place une méthodologie de travail en équipe.
Pour cela nous avons commencé par choisir nos outils de travail.

Pour la gestion des sources, nous avons opté pour le gestionnaire de versions Git qui est pour nous le gestionnaire le plus performant et le plus adapté à nos besoins.
Pour le gestionnaire de projet, nous avons hésité entre utiliser github qui est le gestionnaire de projet utilisé pour le projet TacOS.
Cela permettait de facilement intégrer notre projet à TacOS tout en créant un projet à part.
Seulement, nous avons estimé que github n'était pas assez complet, il lui manquait notamment un bon gestionnaire de tickets.
Nous avons donc opté pour redmine hébergé sur le serveur étudiant de l'INSA.

Nous avons donc utilisé redmine pour la répartition des tâches.
Nous nous sommes également régulièrement réunis pour discuter de l'avancement du projet et des prochaines étapes.
Au début du projet, nous avons estimé les différentes grandes étapes du projet.
Nous avons donc créé plusieurs versions.
Les versions sont très bien gérées par redmine.
Il est possible de créer des versions avec des objectifs et créer des tickets pour chaque version.
On peut donc suivre l'avancement d'une version en fonction de l'avancement des tickets qui lui sont attribués.

Avant de commencer, voici les différentes versions que nous avions estimées :

\begin{itemize}
  \item 0.1
    \begin{itemize}
      \item Création de l'émulateur
      \item Création/conception de pron et la pronlib avec :
        \begin{itemize}
          \item Connexion d'un client
          \item Création d'une fenêtre
          \item Dessin de lignes dans la fenêtre par le client
        \end{itemize}
  \end{itemize}
  \item 0.2
    \begin{itemize}
      \item Mise en place des événements (création de fenêtres)
      \item Début mise en place du gestionnaire de fenêtres Guacamole qui déplace une fenêtre lors de sa création
  \end{itemize}
  \item 0.3
    \begin{itemize}
      \item Amélioration du système de fenêtrage Pron
        \begin{itemize}
          \item Gestion superposition des fenêtres
          \item Création de primitives de dessin (ligne, rectange, ellipses)
          \item Gestion des évènements claviers
      \end{itemize}
  \end{itemize}
  \item 0.4
    \begin{itemize}
      \item Amélioration du système de fenêtrage Pron
        \begin{itemize}
          \item Gestion de la souris et du pointeur de souris
          \item Déplacement, redimensionnement et destruction des fenêtres
          \item Envoi des évènements souris aux clients
      \end{itemize}
      \item Amélioration du gestionnaire de fenêtres Guacamole
        \begin{itemize}
          \item Décoration des fenêtres
          \item Fermeture, maximisation et redimensionnement des fenêtres
      \end{itemize}
  \end{itemize}
  \item 0.5
    \begin{itemize}
      \item Création/conception de la librairie de widgets Sombrero
        \begin{itemize}
          \item Boutons
          \item Labels (textes)
          \item Images
          \item Application
      \end{itemize}
      \item Amélioration du système de fenêtrage Pron
        \begin{itemize}
          \item Gestion du texte
      \end{itemize}
  \end{itemize}
  \item 0.5
    \begin{itemize}
      \item Ajout de widgets avancés dans la librairie de widgets Sombrero
        \begin{itemize}
          \item Layout de grille
          \item Textarea
          \item CheckBox/Radiobutton
          \item Application
        \end{itemize}
    \end{itemize}
  \item 1.0
    \begin{itemize}
      \item Intégration de TacosGUI dans TacOS
      \item création d'applications graphiques
        \begin{itemize}
          \item Émulateur de terminal graphique
          \item Logiciel de dessin Paint-like
          \item Visionneuse d'images
          \item Navigateur de fichiers
          \item Panel de gestion de fenêtres et lanceur d'applications
          \item Fond d'écran dynamique
        \end{itemize}
    \end{itemize}
\end{itemize}

L'objectif de la version 1.0 était de pouvoir faire notre présentation orale du projet sous notre propre système de fenêtrage, ce qui a été fait avec succès et sans erreur de segmentation.
