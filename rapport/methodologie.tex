\section{Méthodologie}

\subsection{Enjeux}

Toute la difficulté du développement de TacOS-GUI réside dans le fait que ce projet soit intimement lié à un système d'exploitation en constante évolution.
Une des voies que nous aurions pu suivre est de développer directement pour TacOS, mais plusieurs problèmes découlent de ce choix.

Le premier est que TacOS, pour l'instant, n'implémente pas la totalité de la libC.
De plus, ayant choisi le C++ comme language de programmation pour notre projet, nous aurions dû développer en premier la STL sur TacOS.
Enfin, TacOS-GUI utilisant un système de fenêtrage client/serveur, il nous fallait un système de communication inter-processus (IPC) ``sockets-like'', ce qui impliquait le développement d'une librairie de sockets.

Tous ces prérequis auraient fortement retardé le début du développement de TacOS-GUI, et nous auraient empêché de paralléliser les tâches.
Nous avons donc opté pour la solution vivement conseillée par nos tuteurs : développer un émulateur qui nous permettrait de développer sous Linux, tout en simulant un environnement TacOS.

\subsection{Émulateur TacOS}

L'émulateur a été développé pour Linux. Comme spécifié dans l'introduction de cette partie, voici les différents points que nous avons dû implémenter :

\begin{itemize}
  \item Environnement TacOS
  \item Drivers
  \begin{itemize}
    \item VESA
    \item Clavier
    \item Souris 
  \end{itemize}
  \item Sockets (\verb|libtsock|)
  \item LibC TacOS
  \begin{itemize}
    \item exec\_elf
  \end{itemize}
\end{itemize}

\subsubsection{Environnement TacOS}

L'environnement TacOS est décrit dans la librairie du même nom, \verb|libtacos|.

La libC de TacOS définissant un certain nombre de fonctions au nom identique à celles de la libC Linux, il nous a fallu surcharger ces fonctions pour leur associer le même comportement que sous TacOS.

Le cas le plus concret concerne les fonctions de manipulation des fichiers : open, read, write, close, ioctl...
\lstinputlisting{listings/open.c}

\subsubsection{Drivers}

L'émulation des différents drivers cités plus haut se fait par le biais d'une librairie graphique, la SDL, au dessus de X.org.
Le driver souris est implémenté grâce aux évènements souris de la SDL, les évènements clavier fonctionnent de la même manière.

Pour ce qui est du driver VESA, la SDL étant une librairie graphique, elle permet de créer un fenêtre X qui représentera notre écran physique. 

\subsubsection{Librairie de sockets}

Au moment du développement de la librairie de sockets \verb|libtsock|, il n'y avait pas d'IPC de ce type développée sous TacOS, nous avons donc pu développer la nôtre sans contrainte particulière, en commençant par une implémentation sur l'émulateur.

Cette première implémentation utilise les sockets UNIX domain. L'implémentation TacOS utilise des buffers alloués en espace noyau, qui représentent la file de messages de chaque socket.

Voici la liste exhaustive des primitives de la librairie de sockets que nous avons défini.

\lstinputlisting{listings/tsock.c}

\subsubsection{LibC TacOS}

En plus de la surcharge des fonctions de la libC Linux, nous avons dû définir les fonctions spécifiques à la libC TacOS. Exemple avec \verb|exec_elf| :
\lstinputlisting{listings/exec_elf.c}
