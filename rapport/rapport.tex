\documentclass[10pt]{article}
\usepackage[utf8]{inputenc}
\usepackage[francais]{babel}
\usepackage{geometry}
\usepackage{pgf}
\usepackage{tikz}
\usepackage{lscape}
\usepackage{pdflscape}
\usepackage{pstricks}
\usepackage{pstcol,pst-fill,pst-grad}
\usepackage{fancyhdr}
\usepackage{hyperref}
\usepackage{float}
\usepackage{listings}
\usepackage{listingsutf8}
% Tikz pour l'affichage d'arbres
\usepackage{tikz}
\usetikzlibrary{trees}
\tikzstyle{every node}=[draw=black,thick,anchor=west]
\tikzstyle{root}=[fill=blue!20]
\tikzstyle{selected}=[draw=red,fill=red!30]
\tikzstyle{optional}=[dashed,fill=gray!50]


\hypersetup{
  colorlinks,
  citecolor=black,
  filecolor=black,
  linkcolor=black,
  urlcolor=black
}
\definecolor{bgblue}{rgb}{0.41961,0.80784,0.80784}%
\definecolor{bgred}{rgb}{1,0.61569,0.61569}%
\definecolor{fgblue}{rgb}{0,0,0.6}%
\definecolor{fgred}{rgb}{0.6,0,0}%
\definecolor{dkgreen}{rgb}{0,0.6,0}
\definecolor{gray}{rgb}{0.5,0.5,0.5}
\definecolor{mauve}{rgb}{0.58,0,0.82}

\lstset{ %
  language=C++,                   % the language of the code
  basicstyle=\footnotesize,       % the size of the fonts that are used for the code
  numbers=left,                   % where to put the line-numbers
  numberstyle=\tiny\color{gray},  % the style that is used for the line-numbers
  stepnumber=1,                   % the step between two line-numbers. If it's 1, each line 
                                              % will be numbered
  numbersep=5pt,                  % how far the line-numbers are from the code
  backgroundcolor=\color{white},  % choose the background color. You must add \usepackage{color}
  showspaces=false,               % show spaces adding particular underscores
  showstringspaces=false,         % underline spaces within strings
  showtabs=false,                 % show tabs within strings adding particular underscores
  frame=single,                   % adds a frame around the code
  rulecolor=\color{black},        % if not set, the frame-color may be changed on line-breaks within not-black text (e.g. commens (green here))
  tabsize=2,                      % sets default tabsize to 2 spaces
  captionpos=b,                   % sets the caption-position to bottom
  breaklines=true,                % sets automatic line breaking
  breakatwhitespace=false,        % sets if automatic breaks should only happen at whitespace
  title=\lstname,                 % show the filename of files included with \lstinputlisting;
                                                                      % also try caption instead of title
  keywordstyle=\color{blue},      % keyword style
  commentstyle=\color{dkgreen},   % comment style
  stringstyle=\color{mauve},      % string literal style
  escapeinside={\%*}{*)},         % if you want to add a comment within your code
  morekeywords={*,vector, iterator} % if you want to add more keywords to the set
}

\geometry{ hmargin=2cm, vmargin=2cm}

\def\blurb{%
Rapport projet tutoré Tacos-GUI}

\def\abstract{%
  Ce document comprends toutes les informations nécessaires pour comprendre l'architecture du projet Tacos-GUI.

Tacos-GUI est un projet de recherche tuteuré effectué par quatre étudiants en 4ème année de l'INSA de Toulouse.
Il consiste à créer une interface graphique pour le système d'exploitation TacOS.
Avant d'entammer la partie technique, nous avons effectué une recherche de ce qui existait déjà dont a découlé un premier document que nous conseillons de lire avant celui-ci.
}

\def\ligne#1{%
  \hbox to \hsize{%
\vbox{\centering #1}}}%

\makeatletter
\def\maketitle{%
  \null
  \vfill
  \vbox{\centering \Large \textbf{\blurb}}
  \vspace{15mm}
  \vbox{\centering \LARGE \textbf{\@title}}
  \vspace{15mm}
  \vbox{\centering \@author}
  \vspace{8mm}
  \vbox{\centering \@date}
  \vspace{15mm}
  \vbox{\centering \textbf{Résumé du projet}}
  \vspace{5mm}
  \vbox{%
    \setlength{\fboxsep}{10pt}
    \centering \fbox{%
      \begin{minipage}{0.9\textwidth}
        \setlength{\parindent}{1cm}
        \setlength{\parskip}{2ex plus .4ex minus .4ex}
        \abstract
      \end{minipage}%
    }
%
  }
  \vfill
  \clearpage
}

\title{Développement d'une interface graphique pour TacOS}
\author{Nicolas Hug \texttt{<hug@etud.insa-toulouse.fr>}\\
  Julien Marchand \texttt{<jmarchan@etud.insa-toulouse.fr>}\\
  Benoît Morgan \texttt{<morgan@etud.insa-toulouse.fr>}\\
Adrien Sulpice \texttt{<sulpice@etud.insa-toulouse.fr>}}
\date{4ème année Informatique}

\begin{document}

\pagestyle{empty}
\setlength{\parskip}{.6ex plus .4ex minus .4ex}
\renewcommand{\headrulewidth}{0pt}
\renewcommand{\footrulewidth}{0.6pt}
\fancyhf{}
\fancyhead{} 
\lfoot{\blurb}
\rfoot{Page \thepage}
\maketitle \clearpage

\setlength{\parskip}{1.2ex plus .4ex minus .4ex}
\pagestyle{fancyplain}

\tableofcontents \clearpage

\subsection{Architecture générale du projet}
\frame{
  \frametitle{4 couches}

  \begin{itemize}
    \item lol
    \item lol
    \item lol
    \item lol
  \end{itemize}
}


\clearpage

\subsection{Travail en groupe}

\paragraph{}
Afin d'optimiser notre travail, nous nous répartissons le travail. Chacun sait donc ce qu'il doit faire de son coté. Cependant il est sans arrêts nécessaire de communiquer. C'est pourquoi nous utilisons des outils que nous décrirons plus tard, nous restons disponibles sur Jabber pour discuter ensemble des problèmes que nous rencontrons, mais surtout, nous faisons des réunions de groupes régulièrement. Ces réunions sont le plus souvent pendant les créneaux prévus dans l'emploi du temps. Seulement ces créneaux ne nous arrangent pas forcément donc des fois nous le faisons à d'autres moments de la semaine si besoin est.

\paragraph{}
Pour éviter de travailler seul chacun dans son coin, nous avons créé deux binômes dans le groupe qui vérifient le travail de l'autre. Ceci permet d'avoir les avantages du travail de groupe tout en restant en très petit commité ce qui permet une avancée rapide. Ces revues sont gérées à l'aide d'outils que nous présenterons plus tard.

\subsection{Outils de travail}

\paragraph{}
Nous avons décidé d'utiliser plusieurs outils pour optimiser le travail en groupe. Avant de commencer à travailler, nous avons découpé le projet en plusieurs versions. Chaque version est numérotée et se voit assigné plusieurs fonctionnalités ainsi qu'une date butoire. Une fois toutes ces fonctionnalités terminées, la version est sauvegardée et nous commençons à travailler sur la suivant. Pour gérer ce système de version et de fonctionnalités, nous utilisons redmine qui est un outil adapté à nos besoins. Il permet la gestion des demandes et l'assignation à un membre de l'équipe. Nous utilisons avec redmine le gestionnaire de version Git qui pour nous est le plus pratique à l'heure actuelle notamment grâce à sa simplicité de gestion des branches. Pour gérer l'avancée du projet, nous avons suivi une méthodologie très structurée. Nous avons une branche master qui est la branche stable du projet. Nous avons ensuite une version de developpement. Nous ne travaillons pas directement sur cette branche. A chaque fois que nous souhaitons ajouter une fonctionnalité nécessitant plus d'un commit, nous créons une nouvelle branche pour cette fonctionnalité. Une fois celle-ci mise en place, nous envoyons les modifications sur la branche de développement. Dès que toutes les fonctionnalités d'une version sont terminées, nous envoyons les modifications sur la branche principale que nous taggons du numéro de version.


\clearpage

\subsection{pron}
\frame{
  \frametitle{pron}

  \begin{block}{LOL}
    \begin{itemize}
      \item va y'avoir des trucs à dire
    \end{itemize}
  \end{block}
}


\clearpage

\subsection{Qu'est-ce que Guacamole ?}
\frame{
  \frametitle{Qu'est-ce que Guacamole ?}

  \begin{itemize}
    \item Gestionnaire de fenêtres dernière génération
    \item Ajout des décorations des fenêtres
    \item Gestion plein écran et maximisation
  \end{itemize}
  \begin{figure}[H]
    \centering
    \includegraphics[width=5cm]{images/guacamole_decoration.png}
    \caption{Décoration Guacamole}
    \label{fig:Guacamole_decoration}
  \end{figure}

}
\subsection{Fonctionnement de Guacamole}
\frame{
  \frametitle{Fonctionnement de Guacamole}
  \begin{figure}[H]
    \centering
    \includegraphics[width=10cm]{images/Guacamole_anim_1.jpg}
    \caption{Fonctionnement Guacamole}
    \label{fig:Guacamole_anim_1}
  \end{figure}
}
\frame{
  \frametitle{Fonctionnement de Guacamole}
  \begin{figure}[H]
    \centering
    \includegraphics[width=10cm]{images/Guacamole_anim_2.jpg}
    \caption{Fonctionnement Guacamole}
    \label{fig:Guacamole_anim_2}
  \end{figure}
}
\frame{
  \frametitle{Fonctionnement de Guacamole}

  \begin{figure}[H]
    \centering
    \includegraphics[width=5cm]{images/guacamole_sans_decoration.png}
    \caption{Décoration Guacamole}
    \label{fig:Guacamole_sans_decoration}
  \end{figure}

}
\frame{
  \frametitle{Fonctionnement de Guacamole}
  \begin{figure}[H]
    \centering
    \includegraphics[width=10cm]{images/Guacamole_anim_3.jpg}
    \caption{Fonctionnement Guacamole}
    \label{fig:Guacamole_anim_3}
  \end{figure}
}
\frame{
  \frametitle{Fonctionnement de Guacamole}
  \begin{figure}[H]
    \centering
    \includegraphics[width=10cm]{images/Guacamole_anim_4.jpg}
    \caption{Fonctionnement Guacamole}
    \label{fig:Guacamole_anim_4}
  \end{figure}
}
\frame{
  \frametitle{Fonctionnement de Guacamole}
  \begin{figure}[H]
    \centering
    \includegraphics[width=10cm]{images/Guacamole_anim_5.jpg}
    \caption{Fonctionnement Guacamole}
    \label{fig:Guacamole_anim_5}
  \end{figure}
}
\frame{
  \frametitle{Fonctionnement de Guacamole}
  \begin{figure}[H]
    \centering
    \includegraphics[width=5cm]{images/guacamole_decoration.png}
    \caption{Décoration Guacamole}
    \label{fig:Guacamole_decoration}
  \end{figure}
}


\clearpage

\subsection{sombrero}
\frame{
  \frametitle{sombrero}

  \begin{itemize}
    \item GTK a du souci à se faire
  \end{itemize}
}


\clearpage

\section{Apports à TacOS}

Malgrès notre investissement, nous n'avons pas eu le temps de porter ce projet sur TacOS dans sa version finale.
En effet, le projet dépends de beaucoup de fonctions qui ne sont pas implémentées sous TacOS.
Notamment la STL
\footnote{Librairie standard du C++ contenant vector, map, list, etc.}.
Il a donc été nécessaire d'implémenter ces classes pas nous-même.
Lors de la tentative de portage finale, nous avons remarqué qu'il en manquait encore quelques-unes.
Notamment la liste chainée, nous n'avions implémenté que la liste doublement chaînée.

Pour ce projet, nous avons décidé d'utiliser le C++.
Le C++ était disponible par défaut sous TacOS sauf pour :

\begin{itemize}
  \item La fonction new : il a été nécessaire de l'implémenter en utilisant malloc()
  \item Les fonctions delete : pareil en utilisant free()
  \item Les exceptions : pour gérer les exceptions, il est nécessaire d'utiliser plusieurs librairies demandant beaucoup de dépendances. Nous n'avons pas utilisé les exceptions pour cette raison
\end{itemize}

Voici une liste non exhaustive des apports de TacOS-GUI à TacOS :
  
\begin{itemize}
  \item STL (C++)
  \item libc
    \begin{itemize}
      \item qsort
      \item gettimeofday
    \end{itemize}
  \item driver
    \begin{itemize}
      \item vesa
      \item vga
      \item souris
    \end{itemize}
\end{itemize}

Une fois que le portage aura été effectué, ce projet apportera plusieurs logiciels graphiques à TacOS.


\clearpage

\section{Conclusion}

Ce projet aura été très formateur sur beaucoup de points.
Il nous permis de découvir de nouvelles notions et d'en approfondir d'autres.
Nous avons du résoudre de nombreuses problématiques très intéressantes et pas toujours faciles à apréhender.
Nous avons du prendre de nombreuses décisions pour arriver à finir le projet.

Ce projet a à la fois porté sur des points très bas niveau (drivers, émulateur), des points de conception d'une architecture (protocole de communication entre les différentes applications), et des points très haut niveau (conception d'une librairie de widgets avec signaux et slots, structures de données avec l'implémentation de la STL TacOS).
Ce projet est donc très vaste et tout le monde peut trouver au moins une partie qui l'intéresse à l'intérieur.

Bien que le projet ait bien avancé, il reste encore beaucoup de choses à faire pour la suite.
Nous regrettons notamment de ne pas avoir eu le temps de le porter sous TacOS dans sa dernière version.
Voici une liste non exhaustive des choses auxquelles nous avons pensé :

\begin{itemize}
  \item Réimplémenter Guacamole avec la librairie de Widgets
  \item Implémentation propre de la librairie sigslot, avec mise à disposition de l'émetteur du signal
  \item Fermeture de l'application par Guacamole propre (ajout de messages)
  \item Fournir d'avantage la librairie de widgets (notamment menus)
  \item Ajout de menus au panel
  \item Gestion de la maximisation propre sans cacher la barre de fenêtres (et plus tard de menus)
  \item Gestion propre du pointeur de souris avec possibilité d'une taille variable
  \item Créer de nouvelles applications
  \item Portage sous TacOS
  \item Ajout des exceptions dans TacOS et utilisation dans Pron, Guacamole et Sombrero
\end{itemize}

Enfin, nous tenons à remercier nos deux tuteurs Maxime Chéramy et Pierre-Emmanuel Hladik pour leur soutien durant ce projet ainsi qu'à toute l'équipe de développement de TacOS qui, sans eux, ce projet n'aurait jamais pu exister.


\end{document}
