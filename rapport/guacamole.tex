\section{Guacamole}

Guacamole est notre gestionnaire de fenêtres flottantes.
Nous allons commencer par rappeller rapidement le rôle d'un gestionnaire de fenêtres mais pour avoir plus de détails, il est conseillé de lire le rapport bibliographique.
Le gestionnaire de fenêtres est l'acteur permettant de :

\begin{itemize}
  \item Décorer\footnote{La décoration d'une fenêtre est le cadre qui contiendra le titre de la fenêtre ainsi que des boutons permettant de la manipuler.} les fenêtres
  \item Fermer, déplacer et redimensionner les fenêtres
  \item Gestion des superpositions
  \item Gérer le plein écran et les maximisation des fenêtres
\end{itemize}

\subsection{Création d'une fenêtre par un client}


Dans l'architecture que nous avons mise en place, Guacamole est un client comme tous les autres.
Il n'a pas un \"status\" différent des autres clients.
Il ne fait que s'abonner à des évènements en plus.
Pour faire son travail, il a besoin d'être averti lors de la création d'une fenêtre et lors de sa destruction.
Il s'abonne donc aux évènements correspondants.
Lorsqu'on lance notre système, on a donc Pron et Guacamole qui sont deux entités distinctes comme ci-dessous.

\begin{figure}[H]
  \centering
  \includegraphics[width=8cm]{images/Guacamole_anim_1.jpg}
  \caption{Guacamole et Pron}
  \label{fig:guacamole_anim_1}
\end{figure}

Lorsqu'une application graphique va démarrer, elle va se connecter à Pron et va envoyer un message de demande de création d'une fenêtre.

\begin{figure}[H]
  \centering
  \includegraphics[width=8cm]{images/Guacamole_anim_2.jpg}
  \caption{Demande de création d'une fenêtre}
  \label{fig:guacamole_anim_2}
\end{figure}

Dès la réception de ce message, Pron va créer la fenêtre sans décoration ni rien.
Le nouvel arbre interne des fenêtres de Pron sera le suivant après le lancement de l'application PronCube :

\begin{center}
  \begin{tikzpicture}[
    grow via three points={one child at (0.5,-0.7) and
    two children at (0.5,-0.7) and (0.5,-1.4)},
    edge from parent path={(\tikzparentnode.south) |- (\tikzchildnode.west)}]
    \node [root] {Root Window}
      child { node {PronCube}};
  \end{tikzpicture}
\end{center}

Il va ensuite regarder s'il y a des applications abonnées aux évènements de créations de fenêtres.
Si Guacamole n'a pas été lancé, on a donc une fenêtre non décorée qui apparaît à l'écran.
Le client fonctionne parfaitement et s'affiche comme ceci :

\begin{figure}[H]
  \centering
  \includegraphics[width=8cm]{images/pron_cube_sans_decoration.jpg}
  \caption{Application PronCube sans décoration}
  \label{fig:pron_cube_sans_decoration}
\end{figure}

En revanche, si Guacamole est lancé, Pron le notifie de la création d'une fenêtre.
Il lui envoie donc un message :

\begin{figure}[H]
  \centering
  \includegraphics[width=8cm]{images/Guacamole_anim_3.jpg}
  \caption{Notification de Guacamole de la création d'une fenêtre}
  \label{fig:guacamole_anim_3}
\end{figure}

Ce message contient plusieurs informations intéressantes :

\begin{itemize}
  \item L'identifiant Pron de la nouvelle fenêtre
  \item L'identifiant Pron de la fenêtre parente
  \item Un booléen autorisant ou non la décoration de la fenêtre
\end{itemize}

Une fois ce message reçu, Guacamole va décorer la fenêtre sauf dans deux cas :

\begin{itemize}
  \item Le booléen de demande de décoration est à faux
  \item Le parent de la fenêtre n'est pas la root window de Pron (Pron id = 0)
            \footnote{On rappelle que chaque application peut créer plusieurs fenêtres à l'intérieur de sa propre fenêtre pour créer les divers éléments graphiques (boutons, labels, et tout autre widget). On ne souhaite donc pas qu'un bouton soit décoré.} 
\end{itemize}

Si la fenêtre doit être décorée, Guacamole va ajouter cette fenêtre à sa liste interne de fenêtres décorées.
Il va ensuite créer la fenêtre de décoration.
Cette fenêtre est donc une fenêtre un peu plus grande que la fenêtre du client.
Elle dépasse notament au dessus pour afficher la décoration.
Guacamole envoie donc une demande à Pron :

\begin{figure}[H]
  \centering
  \includegraphics[width=8cm]{images/Guacamole_anim_4.jpg}
  \caption{Création de la fenêtre de décoration}
  \label{fig:guacamole_anim_4}
\end{figure}

Du coté de Pron, le nouvel arbre des fenêtres est donc le suivant :

\begin{center}
  \begin{tikzpicture}[
    grow via three points={one child at (0.5,-0.7) and
    two children at (0.5,-0.7) and (0.5,-1.4)},
    edge from parent path={(\tikzparentnode.south) |- (\tikzchildnode.west)}]
    \node [root] {Root Window}
      child { node {Décoration de PronCube}}
      child { node {PronCube}};
  \end{tikzpicture}
\end{center}

Guacamole définit ensuite la fenêtre de décoration comme fenêtre parente à la fenêtre du client.
Et positionner la fenêtre de l'application à la bonne position, à l'intérieur de la fenêtre de décoration.
Il envoie donc un message à Pron.

\begin{figure}[H]
  \centering
  \includegraphics[width=8cm]{images/Guacamole_anim_5.jpg}
  \caption{Reparenting de la fenêtre de l'application}
  \label{fig:guacamole_anim_5}
\end{figure}

Le nouvel arbre des fenêtres de Pron est donc le suivant :

\begin{center}
  \begin{tikzpicture}[
    grow via three points={one child at (0.5,-0.7) and
    two children at (0.5,-0.7) and (0.5,-1.4)},
    edge from parent path={(\tikzparentnode.south) |- (\tikzchildnode.west)}]
    \node [root] {Root Window}
      child { node {Décoration de PronCube}
        child { node {PronCube}}
      };
  \end{tikzpicture}
\end{center}

Maintenant que la fenêtre de décoration est créée et à la bonne position.
Il reste deux choses à faire :

\begin{itemize}
  \item Créer les boutons de fermeture, maximisation et redimentionnement de la fenêtre
  \item Trouver un emplacement libre et y déplacer la fenêtre
\end{itemize}

Les boutons de fermeture sont donc de nouvelles fenêtres pour Pron auxquelles Guacamole va s'abonner sur divers évènements que nous verrons plus tard.
Ces boutons seront donc dans la fenêtre de décoration qui sera leur fenêtre parente.
Guacamole envoie donc à Pron plusieurs message de création de fenêtre :

\begin{figure}[H]
  \centering
  \includegraphics[width=8cm]{images/Guacamole_anim_6.jpg}
  \caption{Création des boutons par Guacamole}
  \label{fig:guacamole_anim_6}
\end{figure}

Le nouvel arbre de fenêtres de Pron est maintenant :

\begin{center}
  \begin{tikzpicture}[
    grow via three points={one child at (0.5,-0.7) and
    two children at (0.5,-0.7) and (0.5,-1.4)},
    edge from parent path={(\tikzparentnode.south) |- (\tikzchildnode.west)}]
    \node [root] {Root Window}
      child { node {Décoration de PronCube}
        child { node {Bouton de fermeture}}
        child { node {Bouton de maximisation}}
        child { node {Bouton de redimensionnement}}
        child { node {PronCube}}
      };
  \end{tikzpicture}
\end{center}


\subsection{Gestion des fenêtres}