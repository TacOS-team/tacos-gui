\section{Guacamole}

Guacamole est notre gestionnaire de fenêtres flottantes.
Nous allons commencer par rappeller rapidement le rôle d'un gestionnaire de fenêtres mais pour avoir plus de détails, il est conseillé de lire le rapport bibliographique.
Le gestionnaire de fenêtres est l'acteur permettant de :

\begin{itemize}
  \item Décorer\footnote{La décoration d'une fenêtre est le cadre qui contiendra le titre de la fenêtre ainsi que des boutons permettant de la manipuler.} les fenêtres
  \item Fermer, déplacer et redimensionner les fenêtres
  \item Gestion des superpositions
  \item Gérer le plein écran et les maximisation des fenêtres
\end{itemize}

\subsection{Création d'une fenêtre par un client}


Dans l'architecture que nous avons mise en place, Guacamole est un client comme tous les autres.
Il n'a pas un ``statut'' différent des autres clients.
Il ne fait que s'abonner à des évènements en plus.
Pour faire son travail, il a besoin d'être averti lors de la création d'une fenêtre et lors de sa destruction.
Il s'abonne donc aux évènements correspondants.
Lorsqu'on lance notre système, on a donc Pron et Guacamole qui sont deux entités distinctes comme ci-dessous.

\begin{figure}[H]
  \centering
  \includegraphics[width=8cm]{images/Guacamole_anim_1.jpg}
  \caption{Guacamole et Pron}
  \label{fig:guacamole_anim_1}
\end{figure}

Lorsqu'une application graphique va démarrer, elle va se connecter à Pron et va envoyer une requête de création d'une fenêtre.

\begin{figure}[H]
  \centering
  \includegraphics[width=8cm]{images/Guacamole_anim_2.jpg}
  \caption{Demande de création d'une fenêtre}
  \label{fig:guacamole_anim_2}
\end{figure}

Dès la réception de ce message, Pron va créer la fenêtre sans décoration : pour lui, une fenêtre n'est rien d'autre qu'une zone de dessin rectangulaire.
Le nouvel arbre interne des fenêtres de Pron sera le suivant après le lancement de l'application PronCube :

\begin{center}
  \begin{tikzpicture}[
    grow via three points={one child at (0.5,-0.7) and
    two children at (0.5,-0.7) and (0.5,-1.4)},
    edge from parent path={(\tikzparentnode.south) |- (\tikzchildnode.west)}]
    \node [root] {Root Window}
      child { node {PronCube}};
  \end{tikzpicture}
\end{center}

Il va ensuite regarder s'il y a des applications abonnées aux évènements de création de fenêtres.
Si Guacamole n'a pas été lancé, on a donc une fenêtre non décorée qui apparaît à l'écran.
Le client fonctionne parfaitement et s'affiche comme ceci :

\begin{figure}[H]
  \centering
  \includegraphics[width=8cm]{images/pron_cube_sans_decoration.jpg}
  \caption{Application PronCube sans décoration}
  \label{fig:pron_cube_sans_decoration}
\end{figure}

En revanche, si Guacamole est lancé, Pron le notifie de la création d'une fenêtre.
Il lui envoie donc un message :

\begin{figure}[H]
  \centering
  \includegraphics[width=8cm]{images/Guacamole_anim_3.jpg}
  \caption{Notification de Guacamole de la création d'une fenêtre}
  \label{fig:guacamole_anim_3}
\end{figure}

Ce message contient plusieurs informations intéressantes :

\begin{itemize}
  \item L'identifiant Pron de la nouvelle fenêtre
  \item L'identifiant Pron de la fenêtre parente
  \item Un booléen autorisant ou non la décoration de la fenêtre
\end{itemize}

Une fois ce message reçu, Guacamole va décorer la fenêtre sauf dans deux cas :

\begin{itemize}
  \item Le booléen de demande de décoration est à faux
  \item Le parent de la fenêtre n'est pas la root window de Pron (Pron id = 0)
            \footnote{On rappelle que chaque application peut créer plusieurs fenêtres à l'intérieur de sa propre fenêtre pour créer les divers éléments graphiques (boutons, labels, et tout autre widget). On ne souhaite donc pas qu'un bouton soit décoré.} 
\end{itemize}

Si la fenêtre doit être décorée, Guacamole va ajouter cette fenêtre à sa liste interne de fenêtres décorées.
Il va ensuite créer la fenêtre de décoration.
Cette fenêtre est donc une fenêtre un peu plus grande que la fenêtre du client.
Elle dépasse notament au dessus pour afficher la décoration.
Guacamole envoie donc une demande à Pron :

\begin{figure}[H]
  \centering
  \includegraphics[width=8cm]{images/Guacamole_anim_4.jpg}
  \caption{Création de la fenêtre de décoration}
  \label{fig:guacamole_anim_4}
\end{figure}

Du coté de Pron, le nouvel arbre des fenêtres est donc le suivant :

\begin{center}
  \begin{tikzpicture}[
    grow via three points={one child at (0.5,-0.7) and
    two children at (0.5,-0.7) and (0.5,-1.4)},
    edge from parent path={(\tikzparentnode.south) |- (\tikzchildnode.west)}]
    \node [root] {Root Window}
      child { node {Décoration de PronCube}}
      child { node {PronCube}};
  \end{tikzpicture}
\end{center}

Guacamole définit ensuite la fenêtre de décoration comme fenêtre parente à la fenêtre du client.
Et positionner la fenêtre de l'application à la bonne position, à l'intérieur de la fenêtre de décoration.
Il envoie donc un message à Pron.

\begin{figure}[H]
  \centering
  \includegraphics[width=8cm]{images/Guacamole_anim_5.jpg}
  \caption{Reparenting de la fenêtre de l'application}
  \label{fig:guacamole_anim_5}
\end{figure}

Le nouvel arbre des fenêtres de Pron est donc le suivant :

\begin{center}
  \begin{tikzpicture}[
    grow via three points={one child at (0.5,-0.7) and
    two children at (0.5,-0.7) and (0.5,-1.4)},
    edge from parent path={(\tikzparentnode.south) |- (\tikzchildnode.west)}]
    \node [root] {Root Window}
      child { node {Décoration de PronCube}
        child { node {PronCube}}
      };
  \end{tikzpicture}
\end{center}

Maintenant que la fenêtre de décoration est créée et à la bonne position.
Il reste deux choses à faire :

\begin{itemize}
  \item Créer les boutons de fermeture, maximisation et redimentionnement de la fenêtre
  \item Trouver un emplacement libre et y déplacer la fenêtre
\end{itemize}

Les boutons de fermeture sont donc de nouvelles fenêtres pour Pron auxquelles Guacamole va s'abonner sur divers évènements que nous verrons plus tard.
Ces boutons seront donc dans la fenêtre de décoration qui sera leur fenêtre parente.
Guacamole envoie donc à Pron plusieurs message de création de fenêtre :

\begin{figure}[H]
  \centering
  \includegraphics[width=8cm]{images/Guacamole_anim_6.jpg}
  \caption{Création des boutons par Guacamole}
  \label{fig:guacamole_anim_6}
\end{figure}

Le nouvel arbre de fenêtres de Pron est maintenant :

\begin{center}
  \begin{tikzpicture}[
    grow via three points={one child at (0.5,-0.7) and
    two children at (0.5,-0.7) and (0.5,-1.4)},
    edge from parent path={(\tikzparentnode.south) |- (\tikzchildnode.west)}]
    \node [root] {Root Window}
      child { node {Décoration de PronCube}
        child { node {Bouton de fermeture}}
        child { node {Bouton de maximisation}}
        child { node {Bouton de redimensionnement}}
        child { node {PronCube}}
      };
  \end{tikzpicture}
\end{center}

L'affichage pour pron est donc maintenant le suivant :

\begin{figure}[H]
  \centering
  \includegraphics[width=8cm]{images/pron_cube_avec_decoration.jpg}
  \caption{Application PronCube avec décoration}
  \label{fig:pron_cube_avec_decoration}
\end{figure}

On voit bien la décoration au dessus de la fenêtre avec le titre de la fenêtre à gauche, et les deux boutons de maximisation et de fermeture à droite.
Et également le bouton tout en bas à droite
\footnote{Petit carré plus clair en premier plan.}
de l'application pour redimensionner la fenêtre.

\subsection{Gestion des fenêtres}

Maintenant que nos fenêtres sont décorées, nous avons tous les éléments pour manipuler nos fenêtres.
Pour pouvoir réagir aux actions de l'utilisateur, il faut s'abonner à différents évènements.

\subsubsection{Fermeture et maximisation}

La gestion de la fermeture et de la maximisation sont assez simples.
Il suffit de s'abonner aux évènements de clics sur ces boutons.
Pour la fermeture, lorsqu'un clic est détecté, il faut détruire la fenêtre de décoration.
Ceci détruiera également la fenêtre de l'application.
Pron enverra donc à l'application un évènement et elle se terminera proprement.

La maximisation est également faite lors du clic sur le bouton de maximisation.
Pour maximiser la fenêtre, il faut redimensionner la fenêtre de décoration à la taille de la root window et la placer en coordonnées (0,0).
Il faut également redimensionner la fenêtre de l'application et changer le dessin du bouton de maximisation.

\subsubsection{Déplacement et redimensionnement}

Le déplacement et le redimensionnement sont assez particuliers.
Il faut à la fois gérer le clic mais également les mouvements de la souris.
Pour le déplacement, lors du clic sur la barre de décoration, on s'abonne aux évènements de la souris.
On passe ensuite dans un état ``redimensionnement''.
A chaque mouvement de souris, on va déplacer la fenêtre de décoration d'autant de pixels que la souris s'est déplacée.

Lors d'un déplacement d'une fenêtre par Pron, il déplace automatiquement toutes les fenêtres filles.
Il n'est donc pas nécessaire de déplacer toutes les fenêtres une par une.
Seulement la fenêtre racine qui est la fenêtre de décoration.
Pour des problèmes de performances, lors du déplacement, on évite de redessiner la fenêtre à chaque évènement de souris qui sont très nombreux.
Pour cela, au début du déplacement, on unmap la fenêtre de l'application.
Puis lorsque l'utilisateur relâche la souris, on remap la fenêtre qui s'affiche de nouveau une seule fois.

Pour le redimensionnement, c'est exactement le même principe sauf qu'on passe dans l'état ``redimensionnement''.
Une fois le redimensionnement terminé, l'application est remapée et informée qu'elle a été redimensionnée.
C'est ensuite à elle de mettre à jour les tailles de ses fenêtres filles si besoin et de les afficher de nouveau.



\subsection{Implémentation}

A l'heure de l'écriture de ce document, Guacamole est codé avec la Pronlib directement.
Il y a donc la boucle d'évènements codée directement dans la fonction main, les boutons sont codés dans Guacamole.
Il faudrait donc réimplémenter Guacamole avec la librairie de Widgets.
Utiliser les widgets et la classe Application.