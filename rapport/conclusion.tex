\section{Conclusion}

Ce projet aura été très formateur sur beaucoup de points.
Il nous permis de découvir de nouvelles notions et d'en approfondir d'autres.
Nous avons du résoudre de nombreuses problématiques très intéressantes et pas toujours faciles à apréhender.
Nous avons du prendre de nombreuses décisions pour arriver à finir le projet.

Ce projet a à la fois porté sur des points très bas niveau (drivers, émulateur), des points de conception d'une architecture (protocole de communication entre les différentes applications), et des points très haut niveau (conception d'une librairie de widgets avec signaux et slots, structures de données avec l'implémentation de la STL TacOS).
Ce projet est donc très vaste et tout le monde peut trouver au moins une partie qui l'intéresse à l'intérieur.

Bien que le projet ait bien avancé, il reste encore beaucoup de choses à faire pour la suite.
Nous regrettons notamment de ne pas avoir eu le temps de le porter sous TacOS dans sa dernière version.
Voici une liste non exhaustive des choses auxquelles nous avons pensé :

\begin{itemize}
  \item Réimplémenter Guacamole avec la librairie de Widgets
  \item Implémentation propre de la librairie sigslot, avec mise à disposition de l'émetteur du signal
  \item Fermeture de l'application par Guacamole propre (ajout de messages)
  \item Fournir d'avantage la librairie de widgets (notamment menus)
  \item Ajout de menus au panel
  \item Gestion de la maximisation propre sans cacher la barre de fenêtres (et plus tard de menus)
  \item Gestion propre du pointeur de souris avec possibilité d'une taille variable
  \item Créer de nouvelles applications
  \item Portage sous TacOS
  \item Ajout des exceptions dans TacOS et utilisation dans Pron, Guacamole et Sombrero
\end{itemize}

Enfin, nous tenons à remercier nos deux tuteurs Maxime Chéramy et Pierre-Emmanuel Hladik pour leur soutien durant ce projet ainsi qu'à toute l'équipe de développement de TacOS qui, sans eux, ce projet n'aurait jamais pu exister.
